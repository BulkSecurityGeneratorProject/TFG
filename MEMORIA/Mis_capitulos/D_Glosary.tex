\newglossaryentry{fito}
{
    name={Producto fitosanitario},
    text={producto fitosanitario},
    description={De acuerdo con la Organización Mundial de la Salud (OMS), se define al producto fitosanitario como la \gls{sustancia} o mezcla de sustancias destinadas a prevenir la acción de, o destruir directamente, insectos, ácaros, moluscos, roedores, hongos, malas hierbas, bacterias y otras formas de vida animal o vegetal perjudiciales para la salud pública y también para la agricultura. Inclúyase en este ítem los plaguicidas, defoliantes, desecantes y las sustancias reguladoras del crecimiento vegetal o fitorreguladores - \textit{CCT Mendoza} \cite{cricyt}}
}
\newglossaryentry{sustancia}
{
    name={Sustancia, Sustancia activa, Fármaco},
    text={sustancia},
    description={Un fármaco (o sustancia activa) es toda sustancia química purificada utilizada en la prevención, diagnóstico, tratamiento, mitigación y cura de una enfermedad, para evitar la aparición de un proceso fisiológico no deseado o bien para modificar condiciones fisiológicas con fines específicos. En el dominio de aplicación actual, nos referiremos en concreto a aquellos fármacos utilizados en la prevención, diagnóstico, tratamiento, mitigación y cura de enfermedades relacionadas con los productos agrícolas, marinos o alimenticios - \textit{Fármaco, Wikipedia} \cite{farmacowiki}}
}

\newglossaryentry{certifsanit}
{
    name={Certificación fitosanitaria},
    description={Uso de procedimientos fitosanitarios conducentes a la expedición de un certificado fitosanitario [revisado, 1995] - \textit{Glosario de términos fitosanitarios} \cite{glosarioterminosfito}}
}

\newglossaryentry{certificado}
{
    name={Certificado fitosanitario},
    text={certificado},
    description={Documento oficial que atestigua el estatus fitosanitario de cualquier envío sujeto a reglamentaciones fitosanitarias [FAO, 1990] - \textit{Glosario de términos Mapama} \cite{glosarioterminosmapama}}
}

\newglossaryentry{legisfito}
{
    name={Legislación fitosanitaria},
    text={legislación fitosanitaria},
    description={Leyes básicas que conceden la autoridad legal a la organización nacional de protección fitosanitaria a partir de las cuales podrán elaborarse las reglamentaciones fitosanitarias [FAO, 1990; revisado FAO, 1995] - \textit{Glosario de términos fitosanitarios} \cite{glosarioterminosfito}}
}

\newglossaryentry{tratamiento}
{
    name={Tratamiento fitosanitario},
    text={Tratamientos Fitosanitarios},
    description={Procedimiento oficial para matar, inactivar o eliminar plagas o para esterilizarlas o desvitalizarlas [FAO 1990; revisado FAO, 1995; NIMF 15, 2002; NIMF 18, 2003; CIMF, 2005] - \textit{Glosario de términos fitosanitarios} \cite{glosarioterminosfito}}
}





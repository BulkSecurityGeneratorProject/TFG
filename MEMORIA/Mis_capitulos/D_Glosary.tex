\newglossaryentry{moscow}
{
    name={Método MoSCoW.},
    text={MoSCoW},
    description={Técnica de priorización usada en la gestión de proyectos, análisis de negocio y desarrollo de software para llegar a un acuerdo común con los \textit{stakeholders} (integrantes del proyecto) sobre la importancia que se debería dar a cada requisito. El término \textit{MoSCoW} en sí mismo es un acrónimo derivado de la primera letra de cada categoría de priorización: \textit{Must have} (debe tener), \textit{Should have} (debería tener), \textit{Could have} (podría tener) y \textit{Won't have} (no tendrá). \textit{Wikipedia - MoSCoW} \cite{wikimoscow}}
}
\newglossaryentry{kanban}
{
    name={Kanban.},
    text={Kanban},
    description={Un sistema de gestión de proceso visual que le indica qué producir, cuándo producirlo, y cuánto producir.. \textit{Wikipedia - Kanban} \cite{wikikanban}}
}
\newglossaryentry{mapreduce}
{
    name={MapReduce.},
    text={MapReduce},
    description={Modelo de programación para dar soporte a la computación paralela sobre grandes colecciones de datos en grupos de computadoras. Por regla general se utiliza en problemas con datasets de gran tamaño, alcanzando los petabytes de tamaño. \textit{Wikipedia - MapReduce} \cite{wikimapreduce}}
}
\newglossaryentry{fito}
{
    name={Producto fitosanitario.},
    text={producto fitosanitario},
    description={De acuerdo con la Organización Mundial de la Salud (OMS), se define al producto fitosanitario como la \gls{sustancia} o mezcla de sustancias destinadas a prevenir la acción de, o destruir directamente, insectos, ácaros, moluscos, roedores, hongos, malas hierbas, bacterias y otras formas de vida animal o vegetal perjudiciales para la salud pública y también para la agricultura. Inclúyase en este ítem los plaguicidas, defoliantes, desecantes y las sustancias reguladoras del crecimiento vegetal o fitorreguladores. \textit{CCT Mendoza} \cite{cricyt}}
}
\newglossaryentry{sustancia}
{
    name={Sustancia, Sustancia activa, Fármaco.},
    text={sustancia},
    description={Un fármaco (o sustancia activa) es toda sustancia química purificada utilizada en la prevención, diagnóstico, tratamiento, mitigación y cura de una enfermedad, para evitar la aparición de un proceso fisiológico no deseado o bien para modificar condiciones fisiológicas con fines específicos. En el dominio de aplicación actual, nos referiremos en concreto a aquellos fármacos utilizados en la prevención, diagnóstico, tratamiento, mitigación y cura de enfermedades relacionadas con los productos agrícolas, marinos o alimenticios. \textit{Wikipedia - Fármaco} \cite{farmacowiki}}
}

\newglossaryentry{certifsanit}
{
    name={Certificación fitosanitaria.},
    text={Certificación fitosanitaria},
    description={Uso de procedimientos fitosanitarios conducentes a la expedición de un certificado fitosanitario [revisado, 1995]. \textit{Glosario de términos fitosanitarios} \cite{glosarioterminosfito}}
}

\newglossaryentry{certificado}
{
    name={Certificado fitosanitario.},
    text={certificado},
    description={Documento oficial que atestigua el estatus fitosanitario de cualquier envío sujeto a reglamentaciones fitosanitarias [FAO, 1990]. \textit{Glosario de términos Mapama} \cite{glosarioterminosmapama}}
}

\newglossaryentry{legisfito}
{
    name={Legislación fitosanitaria.},
    text={legislación fitosanitaria},
    description={Leyes básicas que conceden la autoridad legal a la organización nacional de protección fitosanitaria a partir de las cuales podrán elaborarse las reglamentaciones fitosanitarias [FAO, 1990; revisado FAO, 1995]. \textit{Glosario de términos fitosanitarios} \cite{glosarioterminosfito}}
}

\newglossaryentry{tratamiento}
{
    name={Tratamiento fitosanitario.},
    text={Tratamientos Fitosanitarios},
    description={Procedimiento oficial para matar, inactivar o eliminar plagas o para esterilizarlas o desvitalizarlas [FAO 1990; revisado FAO, 1995; NIMF 15, 2002; NIMF 18, 2003; CIMF, 2005]. \textit{Glosario de términos fitosanitarios} \cite{glosarioterminosfito}}
}

\newglossaryentry{mysql}
{
    name={MySQL},
    text={Microsoft SQL Server},
    description={Sistema de gestión de bases de datos relacional gratuito, libre y publicado bajo una licencia \textit{GNU GPL} \cite{gnuglp}. \textit{Wikipedia - MySQL} \cite{wikimysql}}
}

\newglossaryentry{mongodb}
{
    name={MongoDB.},
    text={MongoDB},
    description={Base de datos NoSQL gratis, libre y multiplataforma orientado a documentos (formato JSON) con un esquema. \textit{Wikipedia - MongoDB} \cite{wikimongodb}}
}

\newglossaryentry{cassandra}
{
	name={Apache Cassandra.},
    text={Apache Cassandra},
    description={Sistema de gestión de bases de datos distribuidas NoSQL gratis y libre diseñada para gestionar grandes cantidades de datos a través de diferentes servidores. \textit{Wikipedia - Apache Cassandra} \cite{wikicassandra}}
}

\newglossaryentry{h2}
{
    name={H2 Database Engine.},
    text={H2 Database Engine},
    description={Sistema de gestión de bases de datos relacionales escrito en Java. Puede ser embebido en aplicaciones \textit{Java} o lanzarse en modo cliente-servidor. \textit{Wikipedia - H2 (DBMS)} \cite{wikih2}}
}

\newglossaryentry{bigdata}
{
    name={Big Data.},
    text={Big Data},
    description={Concepto que hace referencia a un conjuntos de datos tan grandes que aplicaciones informáticas tradicionales de procesamiento de datos no son suficientes para tratar con ellos y a los procedimientos usados para encontrar patrones repetitivos dentro de esos datos. {Wikipedia - Big Data} \cite{wikibigdata}}
}
\newglossaryentry{etl}
{
    name={Operaciones ETL.},
    text={ETL},
    description={\textit{Extract, Transform and Load} (extraer, transformar y cargar, frecuentemente abreviado \textit{ETL}) es el proceso que permite a las organizaciones mover datos desde múltiples fuentes, reformatearlos y limpiarlos, y cargarlos en otra base de datos para analizar, o en otro sistema operacional para apoyar un proceso de negocio. {Wikipedia - \textit{Extract, transform and load}} \cite{wikietl}}
}
\newglossaryentry{mariadb}
{
    name={MariaDB.},
    text={MariaDB},
    description={Fork del sistema de gestión de bases de datos relacionales \textit{MySQL} con el objetivo de mantener una versión libre de \textit{MySQL} dada la adquisición del mismo por \textit{Oracle}. \textit{Wikipedia - MariaDB} \cite{wikimariadb}}
}

\newglossaryentry{postgresql}
{
    name={PostgreSQL.},
    name={PostgreSQL},
    description={Sistema de gestión de bases de datos relacional orientado a objetos y libre, publicado bajo la licencia PostgreSQL. \textit{Wikipedia - PostgreSQL} \cite{wikipostgresql}}
}

\newglossaryentry{mssql}
{
    name={Microsoft SQL Server.},
    text={Microsoft SQL Server},
    description={Sistema de gestión de bases de datos relacional desarrollado por \textit{Microsoft}. \textit{Wikipedia - Microsoft SQL Server} \cite{wikimssql}}
}

\newglossaryentry{oracle}
{
    name={Oracle Database.},
    text={Oracle Database},
    description={Sistema de gestión de bases de datos relacional orientado a objetos producido y desarrollado por \textit{Oracle Corporation} \cite{oraclecorporation}. \textit{Wikipedia - Oracle Database} \cite{wikioracle}}
}






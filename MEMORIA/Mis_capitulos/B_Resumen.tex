\begin{center}

{\Large \bfseries RESUMEN EJECUTIVO}

\vspace{1.5cm}
\end{center}


Este proyecto se centra en presentar y demostrar un modelo de integración de información relativa a productos fitosanitarios y su aplicación en diferentes productos agrícolas provenientes de fuentes heterogéneas en un esquema único y escalable. \par
Para lograr esto, se han seleccionado diferentes fuentes de datos sobre productos fitosanitarios, presentes en distintos formatos y se han analizado varias tecnologías de integración para elegir el stack tecnológico que más se adecue al problema en cuestión. Dada la naturaleza del mismo, se puede englobar dentro de un panorama de programación y tecnologías \textit{Big Data}.
\par
Así pues, en este proyecto se han usado herramientas de almacenamiento elegidas bajo un criterio de escalabilidad futura y herramientas de procesado de datos bajo el criterio de proporcionar soporte al mayor abanico de fuentes heterogéneas posibles. El proyecto emplea una base de datos no relacional como sistema responsable del almacenamiento gracias a sus características de escalabilidad, en conjunto con una herramienta capaz de acceder a los datos almacenados a través de una aproximación relacional. Para el procesado de los datos se ha hecho uso de una herramienta que ofrece soporte al procesado de ficheros provenientes de diferentes fuentes, entre las que se encuentra la base de datos elegida, mencionada en este apartado.\par
Para demostrar la viabilidad del sistema, se ha desarrollado un prototipo funcional que recoge datos sobre productos fitosanitarios a nivel de España y Europa e integra la información de ambas fuentes, constituyendo un primer paso hacia ese modelo compartido donde varias fuentes heterogéneas concuerdan en un mismo esquema congruente. Los datos se pueden ver mediante una aplicación web desarrollada con un generador de proyectos ligero sobre \textit{Java} capaz de desplegar rápidamente una aplicación con una cuidada \textit{GUI}.  \\\par

\textbf{Palabras clave:} Integración, Big Data, Productos fitosanitarios, Procesado, Base de datos no relacional, Modelo único, Escalabilidad, Fuentes heterogéneas.

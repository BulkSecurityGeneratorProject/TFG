\chapter{Introducción}
\pagenumbering{arabic}
Actualmente, el proceso de consulta del uso y aplicación de los productos fitosanitarios sobre determinados productos resulta una tarea en ocasiones tediosa, sobre todo, cuando se trata de comprobar las especificaciones y regulaciones que imponen diferentes países en operaciones de importación o exportación de determinados productos agrícolas. Hoy en día, una persona que quiere comercializar estas sustancias debe tener en cuenta varios factores; en primer lugar, existen varios manuales extensos que recogen medidas de seguridad, buenas prácticas y pautas sobre la aplicación de los productos fitosanitarios. Dichos manuales se deben cumplir en todo momento [manual seguridad, aplicación fitosanitarios, buenas prácticas]. No solo eso, sino que por otra parte, las bases de datos o almacenes que recogen la información sobre sustancias autorizadas en muchas ocasiones no están bien gestionadas, son difíciles de encontrar, la información se presenta en formatos heterogéneos e incluso se puede encontrar desactualizada. Los avances conseguidos en este Trabajo de Fin de Grado, pretenden reducir la complejidad de esa tarea de búsqueda de información acerca de productos fitosanitarios, facilitando a una persona el acceso a un esquema común con toda la información centralizada y actualizada.\par

El objetivo principal de este proyecto es \textbf{proponer y validar un proceso de recogida, transformación y presentación de la información sobre productos fitosanitarios con el resultado en forma de esquema compartido estandarizable} que pueda beneficiar tanto a agricultores como a instituciones nacionales o internacionales, \textbf{y facilitar la consulta de dicha información de manera más rápida, simple y accesible que los métodos actuales}. \par
Entre los retos planteados figuran:
\begin{itemize}
\item Desarrollar un sistema capaz de visualizar los datos ya integrados nutriéndose únicamente de las fuentes originales sin intervención de una persona en el proceso
\item La consistencia de los datos y su almacenamiento tanto en formato original como en su formato procesado e integrado en la versión final
\item El diseño de una solución escalable y actualizada en todo momento
\item La posibilidad de añadir características de trazabilidad y mantenimiento a la aplicación.
\end{itemize}

\section{Estructura del documento}
	
Este documento se presenta dividido en varios bloques conceptuales: \par
El primero abarca los primeros tres apartados (Resumen ejecutivo, Introducción  y Definiciones) y se corresponde a una introducción al trabajo realizado; en él, se ofrece una visión completa y resumida del problema junto con su solución y se dan algunas definiciones técnicas de algunos de los términos empleados en esta memoria. \par
El bloque de análisis abarca los apartados 4 y 5 (Phytoscheme y Análisis) y presentan el trabajo de análisis que se llevó a cabo, desde el estudio del entorno, hasta el análisis del stack tecnológico, pasando por el de riesgos y la captura de requisitos. \par 
El tercer bloque de esta memoria se corresponde al bloque de la solución desarrollada en sí; Abarca los apartados 6 y 7 (Diseño e implementación) y en ellos se plasma el trabajo desde las fases tempranas del desarrollo de la solución hasta el momento de la finalización del software como solución tecnológica al problema. \par
Los últimos bloques se corresponden a los detalles de gestión del proyecto, a las conclusiones tanto del proyecto como del alumno a nivel personal y a los diferentes anexos recogidos durante toda la duración del TFG.
